\documentclass[11pt, letterpaper, titlepage]{article}
\usepackage[utf8]{inputenc}
\usepackage{hyperref}
\hypersetup{pdfborder=0 0 0}
\usepackage{amsmath}
\usepackage{amssymb}
\usepackage{geometry}
\usepackage[none]{hyphenat}
\usepackage{xcolor}
\usepackage{cite}
\usepackage{lipsum}
\usepackage{physics}
\usepackage{textgreek}
\usepackage{unicode-math}
\usepackage{subfigure}
\usepackage{graphicx}
\usepackage{textgreek}
\geometry{
 a4paper,
 left=25mm,%left=20mm,
 right=25mm,%right=20mm,
 bottom=25mm, %bottom=20mm,
 top=25mm,%top=20mm,
 }

%%%%%%%% PORTADA
\title{
 \textbf{\LARGE UNIVERSITY OF OSLO} \\
\vspace{37mm}
\textbf{\Large Project report title}\\
\vspace{7mm}
\Large Special Curriculum \\
\vspace{25mm}
} 
\author{\Large Hishem Kløvnes \\ \textcolor{blue}{\href{https://github.com/hishemok/Special_curriculum}{GitHub Link} }} 
\date{\Large \today} % Deadline


\begin{document}
\maketitle
% \tableofcontents
\newpage



\Large Temporary report structure
\section{Introduction}
% Quick introduction of the promise of quantum computing and the major hurdle of decoherence. This is where I introduce topological quantum computation as a potential solution, highlighting its reliance on non-local, protected qubits. 
% \par
% \textbf{Goal:} Realizing MZMs\\
% \textbf{Challenge:} Mention the top-down approach using nanowires and its experimental difficulties (notes from note section 5.1)\\
% \textbf{Alternative:} Introduce the bottom-up approach QD-SC-QD systems to create a minimal, highly tunable Kitaev chain. Define the concept of PMMs. \\
% \textbf{Research Question:} Objective: "This report will provide the theoretical background for superconductivity in hybrid systems and present an analytical and numerical investigation into the emergence and robustness of PMMs in a double QD system". \\
% \textbf{Draft:}\\
Quantum computing holds the promise of revolutionizing information processing by leveraging quantum superposition and entanglement to perform computations far beyond the capabilities of classical computers. A central challenge, howeverm is the fragility of quantum states, which are prone to decoherence from environmental interactions. Environmental interactions tend to rapidly destroy the fragile quantum states that encode information, imposing severe limitations on both the coherence times and error rates. Topologial quantum computation addresses this problem by encoding quantum information non-locally in topologically protected states, making them inherently robust against local perturbations. Among the most promising candidates for such topologically protected states are Majorana zero modes (MZMs), which are exotic quasiparticles that can emerge in certain superconducting systems.\\ \\ 
Superconductivity, as described by Bardeen-Cooper-Schrieffer (BCS) theory, arises from the formation of Cooper pairs. These are bound pair of electrons with opposite momentum and spin. The pairing leads to an energy gap in the excitation spectrum, which protects the superconducting state from low-energy perturbations. Whem superconductivity is induced in hybrid systems, such as semiconductor proximitized by superconductors, it can give rise to emergent quasiparticle excitations with exotic properties. Majorana zero modes, are one such excitation that can appear at the ends of one-dimensional topological superconductors. These modes are predicted to obey non-Abelian exchange statistics, which makes them suitable for fault-tolerant quantum computation, meaning: information is stored in the joint parity of spatially separated modes, rendering it insesitive to local perturbations.\\ \\
To identify and manipulate MZMs, it is important to examine their non-local nature and the conditions under which they arise. The quantum information encoded in these modes is associated with fermionic parity rather than local charge, motivating measurements that distinguish between even and odd parity ground states without collapsing the non-local encoding. This conceptual framework directly informs the design and interpretation of experiments in minimal realizations of Majorana systems, where one seeks to reproduce the essential physics in a controlable, highly tunable setup. \\ \\
In this report, we investigate the Poor Man's Majoranas (PMMs) in a double quantum dot-suprcondcutor-quantum dot (QD-SC-QD) system. PMMs provide a bottom-up, minimal Kitaev chain realization, capturing the key features of MZMs while remaining experimentally accessible and numerically tractable. By combining analytical and numerical approaches, we explore the emergence of zero-energy modes, their spatial localization, and the robustness of their degeneracy under variations in system parameters, providing insight into the feasiblity of realizing topologically protected qubits in such minimal systems.\\

\section{Theoretical Background: The superconducting State}
Notes from note section 1 and note section 2 to build necessary background.\par
\subsection{From Electron Attraction to the Energy Gap:} Short summary of Cooper pair formation, BCS theory, and the concept of the superconducting gap.\par
\subsection{The Bogoliubov-de Gennes Formalism:} Introduce the BdG Hamiltonian as the  mathematical tool for describing quasiparticles in systems where superconductivity varies in space (like at an interface). Important for numerical work later.\par
\subsection{Hybrid Systems and the Proximity Effect:} Explain how a semiconductor can inherit superconducting properties from a nearby superconductor. Define Andreev reflection and the formation of ABS as an important phenomena in hybrid systems.\par

\section{Engineering Majorana Modes in a Minimal System}
Connect general theory to specific PMM platform, notes note section 4 and note section 5.2\par
\subsection{The Kitaev Chain Model:} Introduce the Kitaev model Hamiltonian. Explain its key features (p-wave pairing) and how it leads to a topological phase with protected, zero energy Majorana edge states.\par
\subsection{The QD-SC-QD System as an Effective Kitaev Chain:} Explain the mapping: The two QDs are the sites, Elastic cotunneling is the hopping term, CAR is the p-wave pairing. See note section 5.2 for details.\par

\section{Analytical and Numerical Investigation of PMMs}
Present my work and demonstrate how to apply theory to a concrete problem.\par
\subsection{Single particle model:}
\subsubsection{The Model Hamiltonian:} Present the minimal Hamiltonian for the QD-SC-QD system that i have in my notes note section 5.2.2, for instance Eq 3.54 from chap3.pdf. Define all the terms clearly. This will be the Hamiltonian I use.\par
$$  
  H = \begin{pmatrix}
    ϵ_L & t & 0 & Δ \\
    t & ϵ_R & -Δ & 0 \\
    0 & -Δ & -ϵ_L & -t \\
    Δ & 0 & -t & -ϵ_R
  \end{pmatrix}
$$
Where $ϵ_L$ and $ϵ_R$ are the energy levels of the left and right dots, $t$ is the elastic cotunneling amplitude, and $Δ$ is the CAR amplitude. The basis is $(d_L, d_R, d_L^{†}, d_R^{†})$, called the Nambu basis. The operators $d_L^{†}$ and $d_R^{†}$ create an electron in the left and right dot, respectively.\par

\subsubsection{Analytical Work - Finding the Sweet Spot:} Solve the Hamiltonian for the simplest case: two indentical dots with their energy levels at zero ($ϵ_L = ϵ_R = 0$). I can write the $4x4$ matrix in the Nambu basis $Ψ^{†}= (d_L, d_R, d_L^{†}, d_R^{†})$. Show analytically that the condition for having two zero energy solutions (the Majoranas) is at $|t| = |Δ|$.This proves i understand the sweet spot.\par

Solving the eigenvalue problem $HΨ = EΨ$ gives the characteristic polynomial:
$$
\text{det}(H - EI) = \begin{pmatrix}
    ϵ_L-E & t & 0 & Δ \\
    t & ϵ_R-E & -Δ & 0 \\
    0 & -Δ & -ϵ_L-E & -t \\
    Δ & 0 & -t & -ϵ_R-E
\end{pmatrix} =0
$$
$$
= E⁴ -E²a + b=0
$$
where
$$
a = ϵ_L² + ϵ_R² + 2(t² + Δ²)
$$
and 
$$
b = (t² - ϵ_Lϵ_R - Δ²)² 
$$
The quadratic formula for $E²$ gives:
$$
E² = \frac{a ± \sqrt{a² - 4b}}{2}
$$
The discriminant simplifies to:
$$
a² - 4b = [(ϵ_L+ϵ_R)² + 4t²][(ϵ_L - ϵ_R)² + 4Δ²]
$$
and the eigenvalues are $±\sqrt{E²}$. For zero energy solutions, we need $b=0$, which requires:
$$
t² - ϵ_Lϵ_R - Δ² = 0
$$
At $ϵ_L = ϵ_R = 0$, this reduces to the sweet spot condition:
$$|t| = |Δ|$$
With the conditions we obtained from setting $b=0$, $a$ becomes $2(t² + Δ²)=4t²$.\\
With $b=0$, the quadratic formula simplifies to $E²(E²-a)=0$, giving us two solutions:
$$
E² = 0 \quad E² = a = 4t²
$$
Therefore the four eigenvalues are:
$$
E = \{0, 0, 2|t|, -2|t|\}
$$
These eigenvalues have corresponding eigenvectors. \\
For $E=0$, the eigenvectors are:
$$
v_1 =  \begin{pmatrix} 1 \\ 0 \\ 1 \\ 0 \end{pmatrix}, \quad v_2 = \begin{pmatrix} 0 \\ 1 \\ 0 \\ -1 \end{pmatrix}
$$
and for $E=2|t|$, the eigenvectors are:
$$
v_3 =  \begin{pmatrix} -1 \\ 1 \\ 1 \\ 1 \end{pmatrix}, \quad v_4 = \begin{pmatrix} 1\\ 1 \\ -1 \\ 1\end{pmatrix}
$$
The zero-mode eigenvectors $v_1$ and $v_2$ correspond to the Majorana operators. The two zero-energy eigenvectors,
$v_1$ and $v_2$, represent states that are equal superpositions of particle and hole operators on a single dot. In operator form, they correspond to self-conjugate combinations:
$$
γ_1 ∝ d_L + d_L^{†}, \quad γ_2 ∝ d_R - d_R^{†}
$$
which satisfy the Majorana condition $γ = γ^{†}$. These are the Poor Man's Majorana modes. They are spatially separated zero modes, localized entirely on the left and right dots at the sweet spot $|t| = |Δ|$ and $ϵ_L = ϵ_R = 0$.\\
Their degeneracy at zero energy encodes the nonlocal fermionic parity of the two-dot system, which is the resource we ultimately want for quantum information.\\ 
By contrast, the finite-energy solutions 
$$E = ±2|t|, \quad v_3 ∝ (-1, 1, 1, 1)^T, \quad v_4 ∝ (1, 1, -1, 1)^T$$
are delocalized Bogoliubov quasiparticles. They form the first excited states above the ground state and define the energy gap protecting the Majorana subspace. Physically, this gap is crucial: as long as it remains finite, the Majorana modes cannot hybridize with bulk excitations, and the zero-energy subspace is robust against small perturbations. 

\subsubsection{Numerical Simulations - Visualize the Emergence of Majoranas:} Build and diagonalize the Hamiltonian numerically. Band Structure: Plot the two lowest positive energy eigenvalues as a function of a dot's energy $ϵ_L$ (keeping $ϵ_R = 0$) and $t = Δ$. Reproduce the plot from Fig 3.5a in chap3.pdf showing two states remaining at zero energy, showcasing the symmetry. Wavefunctions: At the sweet spot ($t = Δ$ and $ϵ_L = ϵ_R = 0$), find the eigenvectors for the two zero-mode energy states. Plot the magnitude of the components. I should be able to see that one zero mode is localized entirely on the left dot and the other is entirely on the right dot, confirming they are spatially separated Majoranas.\par
\subsubsection{Numerical Work - Probing Protection and Robustness:} Detuning $t$ and $Δ$: Vary $t$ and $Δ$ away from the sweet spot. Plot the lowest energy eigenvalue as I varyt he ratio $\frac{t}{Δ}$ away from 1. This will show that the zero-energy states immediately split and acquire a finite energy gap, demonstrating their lack of full topological protection. Detuning Dot Energies: Vary $ϵ_L$ and $ϵ_R$ away from zero while keeping $t = Δ$. The degeneracy should be lifted quadratically. My results here should quantify the robustness of the PMMs to local perturbations. I can conclude that while they are not fully protected like in a true topological system, the degeneracy is protected against certain local perturbations to first order.\par

\subsection{Many-body model:} (Including the Coulomb interaction $U$)
\subsubsection{The Model Hamiltonian:} Present the many-body Hamiltonian for the QD-SC-QD system, including the Coulomb interaction term $U$. Define all the terms clearly. This will be the Hamiltonian I use.\par
We have to leave the BdG formalism and work in the many-body basis. This is because the BdG formalism is a mean-field, single-particle approach that cannot capture electron-electron interactions like the Coulomb repulsion $U$. The interaction term we use in order to describe the non-local Coulomb interaction betweem the PMMs is: $H_U=U_{LR}n_Ln_R$, where $n_{α}=d_{α}^{†}d_{α}$ is the number operator for dot $α$.\\
The full Hamiltonian for the QD-SC-QD system with Coulomb interaction is:
$$
H = ∑_{α=L,R} ϵ_{α} d_{α}^{†}d_{α} + t (d_L^{†} d_R + d_R^{†} d_L) + Δ (d_R d_L + d_L^{†} d_R^{†})  + U_{LR} n_L n_R
$$
where $ϵ_{α}$ are the energy levels of the left and right dots, $t$ is the elastic cotunneling amplitude, $Δ$ is the CAR amplitude, and $U_{LR}$ is the non-local Coulomb interaction between electrons on the two dots. The basis states for the two-dot system are: $|0,0⟩$, $|1,0⟩$, $|0,1⟩$, $|1,1⟩$, where $|n_L,n_R⟩$ indicates the occupation of the left and right dots.\par
We can now proceed to construct the Hamiltonian matrix in this many-body basis, where it becomes:
$$
H = \begin{pmatrix}
0 & 0 & 0 & Δ \\
0 & ϵ_R & t & 0 \\
0 & t & ϵ_L & 0 \\
Δ & 0 & 0 & ϵ_L + ϵ_R + U_{LR}
\end{pmatrix}
$$
\subsubsection{Analytical Work - Exploring the many body Hamiltonian:} 
The many-body Hamiltonian above is written in the basis order:
$$ |0,0⟩, |0,1⟩, |1,0⟩, |1,1⟩ $$
Due to order and practicality, we can try to rewrite the Hamiltonian in a block-diagonal manner. To do so, we can first investigat the matrix elements we have. The even sector is spanned by the states $|0,0⟩$ and $|1,1⟩$, which correspond to the matrix elements $H_{11}, H_{14}, H_{41}, H_{44}$. The odd sector is spanned by the states $|0,1⟩$ and $|1,0⟩$, which correspond to the matrix elements $H_{22}, H_{23}, H_{32}, H_{33}$.\\
If we reorder the basis as [even, odd] = $[|0,0⟩, |1,1⟩, |0,1⟩, |1,0⟩]$, we obtain:
$$
H = \begin{pmatrix}
0 & Δ & 0 & 0 \\
Δ & ϵ_L + ϵ_R + U_{LR} & 0 & 0 \\
0 & 0 & ϵ_R & t \\
0 & 0 & t & ϵ_L
\end{pmatrix}
$$
This is now block-diagonal, with the even sector in the top-left $2x2$ block and the odd sector in the bottom-right $2x2$ block.\\
In other words, we have:
$$
H = \begin{pmatrix}
H_{even} & 0 \\
0 & H_{odd}
\end{pmatrix}
$$
with:
$$
H_{even} = \begin{pmatrix}
0 & Δ \\
Δ & ϵ_L + ϵ_R + U_{LR}
\end{pmatrix}, \quad
H_{odd} = \begin{pmatrix}
ϵ_R & t \\
t & ϵ_L
\end{pmatrix}
$$
Our next step is to find an expression for the eigenvalues of each sector. For simplicity we will call $S = ϵ_L + ϵ_R + U_{LR}$. Starting with the even sector, we solve the characteristic polynomial:
$$
\text{det}(H_{even} - EI) = \begin{pmatrix}
    -E & Δ \\
    Δ & S - E
\end{pmatrix} =0
$$
This gives us the quadratic equation:
$$
E² - SE + (Δ²) = 0
$$
Using the quadratic formula, we find the eigenvalues for the even sector:
$$E_{even} = \frac{S ± \sqrt{S² + 4Δ²}}{2} = \frac{ϵ_L + ϵ_R + U_{LR} ± \sqrt{(ϵ_L + ϵ_R + U_{LR})² + 4Δ²}}{2}$$
Next, we solve the characteristic polynomial for the odd sector:
$$
\text{det}(H_{odd} - EI) = \begin{pmatrix}
    ϵ_R - E & t \\
    t & ϵ_L - E
\end{pmatrix} =0
$$
This gives us the quadratic equation:
$$(ϵ_R-E)(ϵ_L - E)  - t² = 0$$
$$E² -(ϵ_R-ϵ_L)E + ϵ_R ϵ_L - t² = 0$$
Using the quadratic formula, we find the eigenvalues for the odd sector:
$$E_{odd} = \frac{(ϵ_L + ϵ_R) ± \sqrt{(ϵ_L - ϵ_R)² + 4t²}}{2}$$

\subsubsection{Analytical Work - Finding the Conditions for a Many-body Sweet Spot:}
To find the conditions for having two degenerate ground states (one from each sector), we set the lowest eigenvalue from each sector equal to each other:
$$E_{\text{even,min}} = E_{\text{odd,min}}$$


\subsubsection{How Majorana like?}
This section is per now just notes: Fill in the blanks along the way.\\
$\ket{e}, \ket{o}$ are the ground states of the even and odd sectors, respectively. Find them:\\\\
Three (four) things to look for to see if we found topologically protected Majoranas:\\
1) Degeneracy of ground states: We need $E_{even,min} = E_{odd,min}\quad  δE=0$\\
2-3) With the number operators $n_L$ and $n_R$, we can calculate the expectation values $\bra{e}n_{L(R)}\ket{e}$ and $\bra{o}n_{L(R)}\ket{o}$. The difference in occupation between the two ground states on each dot should be minimal, ideally zero. This indicates that the Majorana modes are non-local and do not localize charge on either dot.\\
4) Find the Majorana polarization $MP = 1$

\section{Discussion and Conclusion}
\subsection{Synthesize my Findings:} Summarize what we learned from the calculations. " The analysis confirmed that zero-energy Majorana modes emerge under fine-tuned conditions... Numerical simulations revealed that this degeneracy is fragile and splits when deviating from the sweet spot, quantifying the limited protection of PMMs."\par
\subsection{Connect to Broader Context:} Briefly discuss how the properties you explored (the energy splitting, wavefunction overlap) would affect potential braiding operations (note section 5.2.3) and what this means for using PMMs in quantum computing.\par
\subsection{Future Directions:} Suggest next steps for a full thesis. This include modeling disorder, investigating longer chains (three or more dots), or numerically simulate braiding protocol.



\end{document}











